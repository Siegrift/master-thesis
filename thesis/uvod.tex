\chapter*{Introduction} % chapter* je necislovana kapitola
\addcontentsline{toc}{chapter}{Introduction} % rucne pridanie do obsahu
\markboth{Introduction}{Introduction} % vyriesenie hlaviciek

DNA sequencing is a great challenge in modern bioinformatics. Ten years ago, it took a lot of time
and millions of dollars to sequence the whole human genome. Modern technologies can nowdays do so
faster and cheaper. These technologies are very precise, but the device is still very big and
expensive. \cite{barbora_bakalarka}

In 2012, scientists at Oxford Nanopore Technologies developed the new portable device
called MinION that can sequence DNA and costs only several thousands. MinION has, however, several
issues. The main problem is the precision of sequencing. This device measures an electric current
flowing through nanopore with DNA molecule inside. The process of translating this data into DNA
sequence is called basecalling and introduces a lot of errors due to disruptions in measured values.

Another problem we face is the length of the scanned sequence which is often billions of characters
long. Using simple algorithms like DTW or Needleman-Wunsch algorithm which have quadratic time
complexity is not possible.

However, the sequences are not just random strings and we can often have some assumptions about
them, which can be used to design a faster algorithm. The algorithm is allowed not to produce an
optimal solution when aligning the sequences. We tolerate a small error as a tradeoff between time
complexity and precision. Also, if we are using sequences from MinION we already introduced some
innacurracy.
