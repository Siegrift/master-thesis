\chapter*{Concepts and definitions} % chapter* je necislovana kapitola
\addcontentsline{toc}{chapter}{Concepts and definitions} % rucne pridanie do obsahu
\markboth{Concepts and definitions}{Concepts and definitions} % vyriesenie hlaviciek
\label{definitions}

\textit{These are the common terms used in this paper that should be understood by the reader before
  reading the paper. Advanced readers may skip this section and start with the first chapter
  (\ref{chapter_intro}).}

% Need to use "enumerate" otherwise linking stops working. See:
% https://texblog.org/2012/03/21/cross-referencing-list-items/
\begin{enumerate}
  \item \label{def:dom_source_sink} \textbf{DOM source and sink} - In the context of XSS, a DOM
        source is the location from which untrusted data is taken by the application (which can be
        controlled by user input) and passed on to the sink (e.g. location, cookies). Sinks are the
        places where untrusted data coming from the sources is actually getting executed resulting
        in DOM XSS (e.g. eval, element.innerHTML) \cite{source_sink_definition}. We might also refer
        to sink as injection sink, because the untrusted value is injected into the sink by the
        attacker.
  \item \label{def:hot_reload} \textbf{Hot module reload} - The concept of live or hot module
        reload (HMR) is that the running application running in development mode is automatically
        restarted after code changes are made. The former restarts the whole app, the latter only
        patches the running application with code changed and preserves the application state.
\end{enumerate}
