\chapter*{Concepts and definitions} % chapter* je necislovana kapitola
\addcontentsline{toc}{chapter}{Concepts and definitions} % rucne pridanie do obsahu
\markboth{Concepts and definitions}{Concepts and definitions} % vyriesenie hlaviciek
\label{definitions}

\textit{These are the common terms used in this paper that are too broad to be explained in specific
  chapters, but should be understood by the reader before reading the paper. Advanced readers may
  skip this section and start with the first chapter (\ref{chapter_intro}).}

% Need to use "enumerate" otherwise linking stops working. See:
% https://texblog.org/2012/03/21/cross-referencing-list-items/
\begin{enumerate}
  \item \label{def:dom} \textbf{DOM} - DOM stands for \emph{Document Object Model} and it is
        the data representation of the objects that comprise the structure and content of a web
        page.
  \item \label{def:dom_source_sink} \textbf{DOM source and sink} - In the context of XSS, a DOM
        source is the location from which untrusted data is taken by the application and passed on
        to the sink, for example \emph{location} or \emph{cookies}. DOM Sinks are the places where
        untrusted data coming from the sources is actually getting executed resulting in DOM XSS,
        for example \emph{eval} or \emph{Element.innerHTML}. \cite{source_sink_definition}. We might
        also refer to a sink as an \emph{injection sink}, because the untrusted value is injected
        into the sink by the attacker.
  \item \label{def:spa} \textbf{SPA} - SPA stands for \emph{Single Page Applications} and it
        is a website which interacts with the user by dynamically rewriting the contents of the site
        instead of loading new pages from the server.
  \item \label{def:ts} \textbf{TypeScript} - TypeScript is a strict syntactical superset of
        JavaScript and adds optional static typing to the language. It is designed for the
        development of large applications and transpiles to JavaScript.
  \item \label{def:hot_reload} \textbf{Hot module reload} - The concept of a \emph{hot module reload
          (HMR)} is that the running application running in development mode is automatically
        restarted after code changes are made. The former restarts the whole app, the latter only
        patches the running application with code changed and preserves the application state.
  \item \label{def:npm} \textbf{npm} - \emph{npm} is a package manager for the JavaScript
        language and JavaScript runtime environment Node.js.
\end{enumerate}
