\chapter*{Concepts and definitions} % chapter* je necislovana kapitola
\addcontentsline{toc}{chapter}{Concepts and definitions} % rucne pridanie do obsahu
\markboth{Concepts and definitions}{Concepts and definitions} % vyriesenie hlaviciek
\label{definitions}

% Need to use "enumerate" otherwise linking stops working. See:
% https://texblog.org/2012/03/21/cross-referencing-list-items/
\begin{enumerate}
  \item \label{def:dom_source_sink} \textbf{DOM source and sink} - In the context of XSS, a DOM
        source is the location from which untrusted data is taken by the application (which can be
        controlled by user input) and passed on to the sink (e.g. location, cookies). Sinks are the
        places where untrusted data coming from the sources is actually getting executed resulting
        in DOM XSS (e.g. eval, element.innerHTML) \cite{source_sink_definition}.
  \item \label{def:hot_reload} \textbf{Hot reload} - Hot reload just displays the code changes
        according to new code changes without restarting the app.
  \item \label{def:dom_xss} \textbf{DOM XSS} - DOM Based XSS is an XSS attack wherein the attack
        payload is executed as a result of modifying the DOM environment in the victim's browser
        used by the original client side script, so that the client side code runs in an
        "unexpected" manner \cite{owasp_dom_xss_def}.
  \item \label{def:csp} \textbf{CSP} - Content Security Policy (CSP) is an added layer of security
        that helps to detect and mitigate certain types of attacks, including XSS and data injection
        attacks \cite{mdn_csp_def}. CSP provides a way for browsers to create safer APIs in a
        backwards compatible manner, since the API has to be opt in explicitely by the application
        server by sending the CSP response header or using the CSP inside the HTML meta tag in the
        response body. If the browser does not support the CSP directive, the directive is ignored
        and standard browser behaviour applies.
\end{enumerate}
