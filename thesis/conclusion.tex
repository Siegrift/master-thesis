\chapter{Conclusion}

% Should be in past tense: https://blog.wordvice.com/video-which-verb-tenses-should-i-use-in-a-research-paper/

We showed the support of Trusted Types in various open source technologies and discussed their
integrations. We support the claims from the empirical research for web frameworks
\cite{tt_web_framework_paper}.

We discussed Trusted Types integrations for various libraries, frameworks and supporting software.
We see a lot of opportunity for further research, integrations and tools to be created to make
Trusted Types usage easier and more widespread.

We implemented a Trusted Types integration into Solid.js together with an example application
written in this framework. We showed how integration can cascade as we needed to implement minor
changes in multiple packages. The final implementation was fairly small and not that difficult. We
also implemented Cypress end to end tests for the application showing that testing is not a problem
with Trusted Types either with a testing plugin we created.

As a next step we would like to merge our Trusted Types Solid.js integration into the framework
itself as it currently lives only on our forked repositories. It would be nice to see the
integration working on multiple real life applications which run also in production. We would like
to see more web platform primitives to make Trusted Types migration simpler, for example via
HTMLSanitizer \cite{tt_report_mid2021}. We would like to contribute with more open source
integrations and tooling and use our knowledge to help other integration authors. All of our work is
open sourced and available for anyone to see. All repositories we used during the research and
implementation are used as submodules in the main repository.

Unfortunately, currently we do not see a strong demand in the open source community for Trusted
Types compliant applications and libaries, but we hope this will gradually improve. We hope that our
work will encourage other people to create more integrations and that together we will make the web
a safe place.
