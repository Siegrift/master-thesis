\documentclass[12pt, twoside]{book}
%\documentclass[12pt, oneside]{book}  % jednostranna tlac
\usepackage[a4paper,top=2.5cm,bottom=2.5cm,left=3.5cm,right=2cm]{geometry}
\usepackage[utf8]{inputenc}
\usepackage[T1]{fontenc}
\usepackage{graphicx}
\usepackage{url}
\usepackage[hidelinks,breaklinks]{hyperref}
\usepackage{float}
%\usepackage[slovak]{babel} % vypnite pre prace v anglictine
\linespread{1.25} % hodnota 1.25 by mala zodpovedat 1.5 riadkovaniu

% CUSTOM THINGS
\newtheorem{theorem}{Definition}[section]
\newtheorem{corollary}{Corollary}[theorem]
\newtheorem{lemma}[theorem]{Lemma}
\usepackage{amsmath}

% -------------------
% --- Definicia zakladnych pojmov
% --- Vyplnte podla vasho zadania
% -------------------
\def\mfrok{2021}
\def\mfnazov{Trusted Types integration into open source frameworks and libraries}
\def\mftyp{Masters Thesis}
\def\mfautor{Emanuel Tesař, Bc.}
\def\mfskolitel{RNDr. Peter Borovanský, PhD.}

%ak mate konzultanta, odkomentujte aj jeho meno na titulnom liste, TODO: maybe add koto
% \def\mfkonzultant{tit. Meno Priezvisko, tit. }  

\def\mfmiesto{Bratislava, \mfrok}

% bioinformatici odkomentujú riadok s dvoma odbormi a iný program
\def\mfodbor{Computer Science}
%\def\mfodbor{Computer Science and Biology} 
\def\program{Computer Science}
%\def\program{ Bioinformatics }

% Ak je školiteľ z FMFI, uvádzate katedru školiteľa, zrejme by mala byť aj na zadaní z AIS2
% Ak máte externého školiteľa, uvádzajte Katedru informatiky 
\def\mfpracovisko{ FMFI.KAI - Department of Applied Informatics }

\begin{document}
\frontmatter


% -------------------
% --- Obalka ------
% -------------------
\thispagestyle{empty}

\begin{center}
  \sc\large
  Comenius University in Bratislava\\
  Faculty of Mathematics, Physics and Informatics

  \vfill

  {\LARGE\mfnazov}\\
  \mftyp
\end{center}

\vfill

{\sc\large
  \noindent \mfrok\\
  \mfautor
}

\cleardoublepage
% --- koniec obalky ----

% -------------------
% --- Titulný list
% -------------------

\thispagestyle{empty}
\noindent

\begin{center}
  \sc
  \large
  Comenius University in Bratislava\\
  Faculty of Mathematics, Physics and Informatics

  \vfill

  {\LARGE\mfnazov}\\
  \mftyp
\end{center}

\vfill

\noindent
\begin{tabular}{ll}
  Study Programme: & \program      \\
  Field of Study:  & \mfodbor      \\
  Department:      & \mfpracovisko \\
  Supervisor:      & \mfskolitel   \\
  % Consultant: & \mfkonzultant \\
\end{tabular}

\vfill


\noindent \mfmiesto\\
\mfautor

\cleardoublepage
% --- Koniec titulnej strany


% -------------------
% --- Zadanie z AIS
% -------------------
% v tlačenej verzii s podpismi zainteresovaných osôb.
% v elektronickej verzii sa zverejňuje zadanie bez podpisov
% v pracach v naglictine anglicke aj slovenske zadanie

\newpage
\thispagestyle{empty}
% TODO: updatni zadania z AISu ked sa praca schvali
% \hspace{-2cm}\includegraphics[width=1.1\textwidth]{images/zadanie}

% \hspace{-2cm}\includegraphics[width=1.1\textwidth]{images/zadanie-en}

% --- Koniec zadania

\frontmatter

% -------------------
%   Poďakovanie - nepovinné
% -------------------
\setcounter{page}{3}
\newpage
~

\vfill
% TODO: acknowledgments
{\bf Acknowledgments:}

% --- Koniec poďakovania

% -------------------
%   Abstrakt - Slovensky
% -------------------
\newpage
\section*{Abstrakt}

% TODO: fill SK abstract

% TODO: fill in most suitable keywords when the thesis is done
\paragraph*{Kľúčové slová: Trusted Types, Web APIs}
% --- Koniec Abstrakt - Slovensky


% -------------------
% --- Abstrakt - Anglicky 
% -------------------
\newpage
\section*{Abstract}

% TODO: fill EN abstract
Trusted Types is a modern Web API which aims to reduce DOM XSS attack surface in web applications.
They give you the tools to write and maintain applications free of DOM XSS vulnerabilities by making
the dangerous web API functions secure by default. Currently, they are supported in Chrome, Edge and
Opera.

Integrating Trusted Types in web application requires code changes. The problem is when these
changes need to be made in third party code which you can't easily modify. Trusted Types support in
open source projects is gradually improving and our plan is to analyze these integrations and
implement one or more of the challenging ones.

% TODO: fill in most suitable keywords when the thesis is done
\paragraph*{Keywords: Trusted Types, Web APIs}

% --- Koniec Abstrakt - Anglicky

% -------------------
% --- Predhovor - v informatike sa zvacsa nepouziva
% -------------------
%\newpage 
%\thispagestyle{empty}
%
%\huge{Predhovor}
%\normalsize
%\newline
%Predhovor je všeobecná informácia o práci, obsahuje hlavnú charakteristiku práce 
%a okolnosti jej vzniku. Autor zdôvodní výber témy, stručne informuje o cieľoch 
%a význame práce, spomenie domáci a zahraničný kontext, komu je práca určená, 
%použité metódy, stav poznania; autor stručne charakterizuje svoj prístup a svoje 
%hľadisko. 
%
% --- Koniec Predhovor


% -------------------
% --- Obsah
% -------------------

\newpage

\tableofcontents

% ---  Koniec Obsahu

% -------------------
% --- Zoznamy tabuliek, obrázkov - nepovinne
% -------------------

\newpage

\listoffigures
\listoftables

% ---  Koniec Zoznamov

\mainmatter


% \input uvod.tex 

% \input dnaSequencing.tex

% \input sequenceAlignment.tex
%\input latex.tex

%\input lorem.tex

%\input zaver.tex

% -------------------
% --- Bibliografia
% -------------------


\newpage

\backmatter

\thispagestyle{empty}
\nocite{*}
\clearpage

\bibliographystyle{plain}
\bibliography{literatura}

%Prípadne môžete napísať literatúru priamo tu
%\begin{thebibliography}{5}

%\bibitem{br1} MOLINA H. G. - ULLMAN J. D. - WIDOM J., 2002, Database Systems, Upper Saddle River : Prentice-Hall, 2002, 1119 s., Pearson International edition, 0-13-098043-9

%\bibitem{br2} MOLINA H. G. - ULLMAN J. D. - WIDOM J., 2000 , Databasse System implementation, New Jersey : Prentice-Hall, 2000, 653s., ???

%\bibitem{br3} ULLMAN J. D. - WIDOM J., 1997, A First Course in Database Systems, New Jersey : Prentice-Hall, 1997, 470s., 

%\bibitem{br4} PREFUSE, 2007, The Prefuse visualization toolkit,  [online] Dostupné na internete: <http://prefuse.org/>

%\bibitem{br5} PREFUSE Forum, Sourceforge - Prefuse Forum,  [online] Dostupné na internete: <http://sourceforge.net/projects/prefuse/>

%\end{thebibliography}

%---koniec Referencii

% -------------------
%--- Prilohy---
% -------------------

%Nepovinná časť prílohy obsahuje materiály, ktoré neboli zaradené priamo  do textu. Každá príloha sa začína na novej strane.
%Zoznam príloh je súčasťou obsahu.
%
%\input appendixA.tex

%\input appendixB.tex

\end{document}
