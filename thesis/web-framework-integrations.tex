\chapter{Integrations into web frameworks}

In this chapter, we describe a partial Trusted Types support for Next.js. Then we show how to use
Trusted Types with React. React already supports Trusted Types behind a feature flag and we only
show how to enable this feature flag for applications created via \emph{Create React App}. Lastly,
we build upon the Solid.js integrations from the previous chapter and verify the integrations in an
open-source real-world application. This shows both, that our integrations are correct and also how
to migrate an application to Trusted Types.

Web frameworks and libraries is a software that is designed to support the development of web
applications and services. They try to solve common problems faced in web development, such as
building user interfaces, testing, building, and bundling the application. The typical web
application consists of numerous libraries and software frameworks that together create the
resulting web application.

\section{Next.js integration}
\label{intro-nextjs}

Next.js is one of the most popular frameworks for building web applications. The framework is built
upon React which is the most used UI framework as of 2021 \cite{react_most_used_2021}. Parts of a
Next.js application can be rendered statically, server-side, or fully client-side on a page-by-page
basis. Next.js is thus not only about the client-side code, but it can also handle server-side logic
which opens up different means of attacks such as reflected XSS, SQL injections and more.

Next.js was our initial choice for Trusted Types integration because of the large impact this
integration would have. The framework itself seemed interested in the integration of Trusted Types
for a longer time \cite{nextjs_tt_pr_2020}.

We started working on the integration and created a basic Next.js application for testing. We used a
local version of Next.js as a dependency for our application. The example application was not
working when Trusted Types were enforced and we needed to make some changes in the Next.js. We ended
up with a working version of the integration which supported the application in development mode
with Trusted Types under enforcement mode. We managed to accomplish this with one simple Trusted
Types policy in a short timeframe.

\bigskip
\begin{lstlisting}[language=JavaScript, caption=Example of Next.js Trusted Types API]
let policy;

const whitelistAll = (str) => str;

// The policy getter is a private part of the module
// and cannot be used directly.
const getOrCreatePolicy = () => {
  if (policy) return policy;

  policy = window.trustedTypes?.createPolicy('next', {
    createHTML: whitelistAll('createHTML'),
    createScript: whitelistAll('createScript'),
    createScriptURL: whitelistAll('createScriptURL'),
  });
  return policy;
};

export const __unsafeAllowHtml = (html) => getOrCreatePolicy()?.createHTML(html) ?? html;

export const __unsafeAllowScriptUrl = (scriptUrl) => getOrCreatePolicy()?.createScriptURL(scriptUrl) ?? scriptUrl;

export const __unsafeAllowScript = (script) => getOrCreatePolicy()?.createScript(script) ?? script;
\end{lstlisting}

The fixes needed were small and the Trusted Types API specific code was encapsulated in a single
small module. That said, the implementation was only a proof of concept. However, we found out that
there are other engineers working on this integration and we decided not pursue this project
further. Instead we shifted our focus to a different projects and integrations.

\subsection{Tsec violations}

Tsec found 8 violations \cite{tsec_output} inside Next.js sources. Out of these 7 were indeed
Trusted Types violations that needed to be fixed. Some of these could be fixed simply on a type
system level since they expected a value from the user. The others needed to be explicitly allowed
through a policy. The implementation for this proof of concept can be found on GitHub
\cite{nextjs_fix_tsec_violations_commit}.

Since our utmost goal was to find the sinks and create a prototype for the integration we wrapped
all of these values in Trusted Types objects. This was to be revisited in the future.

\subsection{Development mode violations}

Fixing the violations found by Tsec was not enough and the application still would not work under
Trusted Types enforcement. This was caused by a Webpack plugin used in one of the Next.js
dependencies which used eval internally. The workaround for this was to use a default policy and
allow all eval calls. The proper solution would be to fix the eval issue in the Webpack plugin via a
policy.

Apart from this, another violation that was triggered was caused by the application hot reloading
during development. Tsec did not catch this problem since the violation came from a JavaScript file
where the AST information was limited.

\section{Create React App integration}

Create React App (CRA) is a CLI and an officially recommended way to create single-page React
applications. It offers an easy React application setup with no configuration. The source code of
this tool does not depend on React directly. It is only used to generate the project files based on
a hard-coded template and then it installs the latest version of React and other necessary
dependencies.

CRA was a second project we wanted to integrate Trusted Types to. Our goal was to make sure the
generated application is Trusted Types compatible.

\medskip
\begin{flushleft}\textbf {Using Trusted Types compatible version of React}\end{flushleft}
\medskip

To accomplish this, we would need to change the implementation of CRA to install the Trusted Types
compatible version of React. Unfortunately, such version of React is implemented only under a
feature flag that needs to be turned on at build time. The published version of React has a feature
flag turned off. This means that to use the Trusted Types compatible version of React, the
developers need to edit the React source code and turn on the feature flag. Then build the framework
themselves. This custom-built version can then be used as a dependency in the project generated by
CRA.

Implementing this is non-trivial since it requires knowledge about React. More importantly, this is
harder to maintain for the application authors since they need to keep up with the new releases of
React manually.

\medskip
\begin{flushleft}\textbf {Using Trusted Types compliant version of Webpack}\end{flushleft}
\medskip

CRA internally uses Webpack 5 to provide convenient development features, transform and bundle
production applications. Some of these features cause Trusted Types violations. Gladly, Webpack can
be configured to be Trusted Types compliant by a small configuration change
\cite{webpack_tt_config}.

The problem with CRA is that the Webpack configuration is hidden from the user. Unfortunately, there
is no way how to override this configuration manually. The workaround is to spy on imported
JavaScript modules and change their content. This pattern is often used when mocking or spying in
unit tests. However, it is also a suitable solution for this case \cite{cra_modify_webpack_config}.

Another issue we faced was caused by \emph{webpack-dev-server} which is only used in the development
and provides features like hot module replacement or showing error overlays. We found out that the
error overlay widget is created via \emph{innerHTML} which causes a Trusted Types violation and
prevents the error from being shown. The solution is to create a policy for the overlay widget and
use a custom version of this dependency. Since Trusted Types are supported in Webpack we directly
opened a GitHub issue \cite{tt_webpack_dev_server:issue} and proposed our solution
\cite{tt_webpack_dev_server:pr}.

\bigskip
\begin{lstlisting}[language=JavaScript, caption=Script to start React application with Trusted Types enabled in Webpack]
// File 'scripts/start.js'
const rewire = require('rewire')
const defaults = rewire('react-scripts/scripts/start.js')
const webpackConfig = require('react-scripts/config/webpack.config')

// In order to override the Webpack configuration without ejecting the create-react-app
defaults.__set__('configFactory', (webpackEnv) => {
  let config = webpackConfig(webpackEnv)

  // Customize the Webpack configuration here, for reference I have updated Webpack externals field
  config.output.trustedTypes = {
    policyName: 'webpack-policy',
  }

  return config
})
\end{lstlisting}

The application is then started simply by running this script using the Node.js environment.

\bigskip
\begin{lstlisting}[language=JavaScript, caption=Starting the CRA application]
node ./scripts/start
\end{lstlisting}

Another option is to use the \emph{eject} command from CRA. This is a one-way operation and it
unwraps all of the hidden configurations and creates these files in the project. Developers can then
edit the exposed Webpack configuration which is now part of the project. This is not recommended and
should be used only when other attempts fail. The generated Webpack configuration is pretty complex
and updating it in the future might not be trivial.

\section{Solid.js integration}
\label{section:solid}

Solid.js is a simple, modern, and reactive framework for building user interfaces. Framework
syntax is largely inspired by React, but the internals are different. JSX internally makes use of
HTML template elements. Both frameworks are component-oriented. The main difference is that React
uses a concept called \emph{virtual DOM} in which the UI representation is kept in memory and synced
with the "real" DOM. Solid.js does not use virtual DOM and performs all UI updates directly.

We have already described the Solid.js JSX transformation in the preprocessor chapter
(\ref{jsx_solid_preprocessor}) where we implemented a Trusted Types integration. We wanted to test
the changes on a real-world project. Fortunately, there is a fully fledged full stack application
which we used \cite{original_solid_realworld_project}. We made a few additional changes,
specifically:

\begin{enumerate}
  \item Used custom version of Vite and Solid.js
  \item Added Trusted Types policies
  \item Implemented e2e tests
\end{enumerate}

Finally, we documented all the necessary changes needed to build the project and try the application
with all the integrated projects \cite{solid_realworld_project:readme}.

\subsection{Using custom dependencies}

The real-world project uses the Rollup bundler, but we decided to replace Rollup with Vite because
Vite is the preferred bundler for developing Solid.js applications and we also wanted to test that
our integration works properly.

The only changes we needed to implement were to use our custom version of Vite and rename the file
extension for all source files from \emph{.js} to \emph{.jsx} for Vite to correctly preprocess the
JSX. We also used the custom version of Solid.js to generate Trusted Types compliant code after JSX
transformation. See the necessary code changes in \cite{solid_realworld_project:vite}.

\subsection{Adding Trusted Types policies}

After all of the dependencies were ready, we needed to enable the Trusted Types enforcement. We used
a HTML meta tag and enabled policies needed. Because hot reloading in Vite reloads the full module,
we had to use \emph{'allow-duplicates'} to allow recreating the policy when it's module is reloaded.
Having this is not needed for production.

\bigskip
\begin{lstlisting}[language=JavaScript, caption=Creation of style elements using \emph{innerHTML} in Vite \cite{commit_vite_inner_html_styles}]
<meta
  http-equiv="Content-Security-Policy"
  content="require-trusted-types-for 'script'; trusted-types solid-dom-expressions trusted-article vite-overlay 'allow-duplicates';"
/>
\end{lstlisting}

We also needed to create a Trusted Types policy for the application itself. One of the source files
uses a third-party API to load a HTML content and assign that into \emph{innerHTML} property. This
could easily lead to an XSS if the API was malicious or got hacked. This means that Trusted Types
helped find and prevent a possible attack vector. Since the application used this only for
demonstration purposes we decided to allow this pattern via a policy.

See the necessary changes in \cite{solid_realworld_project:tt_policy}.

\subsection{Implementing end to end tests}

After making sure the application works as intended both in development and production, we decided
to create e2e tests to verify this. We chose Cypress as the testing framework and created our
testing plugin which is described in its chapter (\ref{cypress_testing_plugin}). See the
implementation of these tests in \cite{solid_realworld_project:tests}.
