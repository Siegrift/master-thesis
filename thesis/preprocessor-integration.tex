\chapter{Integrations into preprocessors}

In this chapter, we describe a large group of integrations, formally defined as preprocessors. The
rest of the chapter explains our integrations. The first one is a JSX Babel plugin for the Solid.js
framework. The next one is a Vite integration to support the development mode of applications. These
are the two core integrations that provide the support for the development of Trusted Types
compliant applications in the Solid.js framework.

It is very common for web applications to depend on multiple preprocessing tools which can perform
many different things. For example, one can consider the TypeScript compiler a code preprocessor
that compiles TypeScript code to JavaScript. One of the most notorious code preprocessors is Babel,
which takes a modern JavaScript code and transpiles it into an older versions that are then
compatible with a wide range of browsers and their versions. However, the biggest group of code
preprocessors are bundlers which simplify the developer experience by providing advanced development
features such as hot reloading, debugging and source map support.

These tools are essential for web application developers and are usually configured declaratively
and abstracted away from the application authors. Unfortunately, these transformations might produce
code that is not Trusted Types compliant. To achieve Trusted Types compliance in frameworks and
applications, one must ensure that the resulting code from preprocessing tools is Trusted Types
compliant or implement the Trusted Types integration themselves.

There are a lot of preprocessors that do not affect the Trusted Type compliance of a code. It is
hard to formally define this group, but these are usually the code transformations that do not
change the semantics of the code. For example:

\begin{itemize}
  \item TypeScript -- TypeScript (\ref{def:ts}) compiler transforms the TypeScript sources to
        equivalent code written in JavaScript by mostly removing the type annotations. Such
        transformation yields semantically equivalent code.
  \item Transformation to older JavaScript standards -- One of the functionalities of Babel is to
        transpile code written in modern JavaScript language to older standards compatible with a
        wide range of browsers and browser versions. Many of these transformations are solved with
        polyfills or simple AST replacements but the code remains semantically the same.
  \item Minification -- The purpose of the code minification is to reduce the overall bundle size
        that needs to be sent by the server to the clients' browsers. In practice, this includes
        removing dead code, changing the names of the functions, shortening expressions and
        statements, however, the minified code is functionally equivalent to the non-minified one.
\end{itemize}

There are many Trusted Types compliant preprocessors that work with Trusted Types out of the box.
Unfortunately, there are also tools and preprocessors which produce code that is not Trusted Types
compliant. For these tools, a Trusted Types integration is needed.

\section{Babel}

Babel is an extremely large collection of tools and plugins for preprocessing source code. For this
paper, the most interesting feature of Babel is JSX transformation. JSX is an XML-like syntax
extension to ECMAScript without any defined semantics. It is intended to be used by preprocessors to
transform the JSX markup into standard JavaScript. These preprocessors provide the semantics the JSX
on its own does not have.

The JSX was originally designed to simplify the syntax for React applications by allowing mixing
HTML, JavaScript, and CSS all in the same file. However, the specification \cite{jsx_spec} is
generic and can be used for multiple use cases and not only strictly UI related.

\medskip
\begin{flushleft}\textbf {Solid.js JSX preprocessor}\end{flushleft}
\medskip
\label{jsx_solid_preprocessor}

One of the JSX use cases is the JSX preprocessing inside the Solid.js framework
(\ref{section:solid}). This framework on the surface looks like React as it is component-oriented,
uses the JSX, and supports native APIs which resemble the React hooks API. The important fact is that
the translated JSX is very different under the hood. React translates the JSX to
\text{React.createElement} calls but Solid.js translates the JSX markup to native HTML templates.
The framework differences are not that important for this paper. However, both of them use Babel to
transform the easier to more convenient JSX syntax into JavaScript compatible code. See the
transformation in action in the code listings below (generated using Solid.js playground
\cite{solidjs_playground}).

\bigskip
\begin{lstlisting}[language=JavaScript, caption=Example of a component in Solid.js using JSX]
import { createSignal, splitProps } from "solid-js";

export function Counter(props) {
  const [local] = splitProps(props, ["ref"]);
  const [count, setCount] = createSignal(0);
  const increment = () => setCount(count() + 1);

  return (
    <button ref={local.ref} type="button" onClick={increment}>
      {count()}
    </button>
  );
}
\end{lstlisting}

\bigskip
\begin{lstlisting}[language=JavaScript, caption=Example of a Solid.js component after Babel transformation]
import { template, delegateEvents, insert } from 'solid-js/web';
import { splitProps, createSignal } from 'solid-js';

const _tmpl$ = template(`<button type="button"></button>`, 2);
function Counter(props) {
  const [local] = splitProps(props, ["ref"]);
  const [count, setCount] = createSignal(0);

  const increment = () => setCount(count() + 1);

  return (() => {
    const _el$ = _tmpl$.cloneNode(true);

    _el$.$$click = increment;
    const _ref$ = local.ref;
    typeof _ref$ === "function" ? _ref$(_el$) : local.ref = _el$;

    insert(_el$, count);

    return _el$;
  })();
}

delegateEvents(["click"]);

export { Counter };
\end{lstlisting}

The transformed Solid.js code uses a \text{template} function (imported from \emph{solid-js/web}).
This function internally uses the \emph{HTMLTemplateElement.innerHTML} property when creating the
template. This triggers a Trusted Types violation when the code is executed in the browser. The
solution is to create a Trusted Types policy wrapping the content assigned to the \emph{innerHTML}
property. This is secure because the JSX transformation only creates the templates from statically
known HTML. All dynamic and interpolated markup is rendered using JavaScript without using any
dangerous sinks.

\bigskip
\begin{lstlisting}[language=JavaScript, caption=Examples of JSX transformations]
// Example input 1
let dynamic = "<img src=x onerror='alert(1)'>"
return (
  <iframe srcdoc={dynamic}></iframe>
);

// Example output 1
const _tmpl$ = template(`<iframe srcdoc="<img src=x onerror='alert(1)'>"></iframe>`, 3);

// Example input 2
let dynamic = "<img src=x onerror='alert(1)'>"
dynamic += '<br />'
return (
  <iframe srcdoc={dynamic}></iframe>
);

// Example output 2
const _tmpl$ = template(`<iframe></iframe>`, 2);

// Example input 3
<form action="/action_page.php">
  <input type="text" id="fname" value={name()} />
  <input type="text" id="lname" value={lname()} />
  <input type="submit" value="Submit" />
</form>

// Example output 3
const _tmpl$ = template(`<form action="/action_page.php"><input type="text" id="fname"><input type="text" id="lname"><input type="submit" value="Submit"></form>`, 5);
\end{lstlisting}

The JSX preprocessing is able to ensure only statically known JSX values are passed to the
\emph{template} function. These statically known values are coming from the source code written by
the developer, so this code is implicitly trusted. Any dynamic properties and attributes are set
when the component renders and is not part of the generated template calls.

\section{Bundlers}

Bundlers deal with many different tasks such as code minification and obfuscation, loading different
file types, bundling the source files into big chunks, hot reloading in development and more.

Many of the bundler responsibilities are Trusted Types compliant out of the box. However, many of
the development only features break when Trusted Types are enforced. Usually, bundlers include a
development server that injects some markup into DOM to provide a better developer experience for
the application authors. Common examples include hot reloading changed parts of the application or
displaying error details in an overlay widget.

Fortunately, Trusted Types are already supported in Webpack (starting from version 5), which is the
most used bundler at the time of writing. However, there are some new promising ones that provide
better performance or use ES modules which leads to simpler architecture and a more performant
design.

\medskip
\begin{flushleft}\textbf {Vite}\end{flushleft}
\medskip

Vite is one of the modern bundlers which serves the source files via native ES modules which have
many built-in features and extremely fast hot module replacement (HMR) for development. It also
provides a simple way to bundle production code using another common bundler called Rollup.

Vite is a preferred bundler for creating applications using the Solid.js framework. To support the
development of these applications, we needed to create a Trusted Types integration for Vite.

We followed the integration process from chapter (\ref{tt_integration_setup}) and have
identified three places that cause Trusted Types violations. These can be categorized into:

\begin{itemize}
  \item Creation of style elements -- Internally, Vite used \emph{innerHTML} of an HTML style
        element to apply styles to the DOM. These two occurrences can be replaced with a
        safe \emph{textContent} property.

        \bigskip
        \begin{lstlisting}[language=JavaScript, caption=Creation of style elements using \emph{innerHTML} in Vite \cite{commit_vite_inner_html_styles}]
if (!style) {
  style = document.createElement('style')
  style.setAttribute('type', 'text/css')
  style.innerHTML = content
  document.head.appendChild(style)
} else {
  style.innerHTML = content
}
\end{lstlisting}

  \item Showing error overlay -- Vite supports HMR during development and part of this feature is to
        display an error overlay when there is an error triggered by the development server. This
        usually happens when there is a syntax error in the code. The overlay widget uses the
        \emph{innerHTML} property to inject the overlay HTML markup on top of the client's web
        page. In this case, the solution was to introduce a Trusted Types policy to wrap the overlay
        code inside the Trusted Types policy. An alternative solution would be to create the overlay
        markup dynamically using \emph{document.createElement}.

        In the future, Trusted Types API might provide support for constructing trusted values from
        string constants via template literals without using a policy, but this is not yet supported
        \cite{tt_from_literal_method}.

        \bigskip
        \begin{lstlisting}[language=JavaScript, caption=Creation of error overlay using \emph{innerHTML} property \cite{commit_vite_inner_html_overlay}]
export class ErrorOverlay extends HTMLElement {
  root: ShadowRoot

  constructor(err: ErrorPayload['err']) {
    super()
    this.root = this.attachShadow({ mode: 'open' })
    this.root.innerHTML = template
    (further lines omitted for brevity...)
\end{lstlisting}

        \bigskip
        \begin{lstlisting}[language=JavaScript, caption=Creation of error overlay using Trusted Types policy \cite{commit_vite_policy_overlay}]
export class ErrorOverlay extends HTMLElement {
  root: ShadowRoot

  constructor(err: ErrorPayload['err']) {
    super()

    let policy
    if (window.trustedTypes) {
      policy = window.trustedTypes.createPolicy('vite-overlay', {
        createHTML: (s) => s
      })
    }

    this.root = this.attachShadow({ mode: 'open' })
    this.root.innerHTML = policy
      ? (policy.createHTML(template) as any)
      : template
    (further lines omitted for brevity...)
\end{lstlisting}


\end{itemize}
