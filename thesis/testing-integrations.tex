\chapter{Integrations into testing frameworks}

In this chapter, we describe a Cypress plugin that we created to simplify the testing of Trusted
Types in web applications. This plugin is created from scratch, verified with many tests, and also
used to test the real-world application written in Solid.js from the previous chapter.

Trusted Types are agnostic to the testing framework used and the type of tests. It is very common
for a web application to have a combination of unit and end-to-end tests. The former are used to
test smaller parts of applications and typically run in the Node.js environment, where neither DOM
nor Trusted Types are available. The latter usually test the whole application in a browser
environment and test the application features from the user perspective.

\section{Cypress Trusted Types plugin}
\label{cypress_testing_plugin}

Cypress is one of the most popular frameworks for end-to-end testing of web applications
\cite{js_state:testing}. It enjoys a rich ecosystem of plugins and supporting software. It runs the
tests in real browsers. Developers can test Trusted Types compliant applications in Cypress out of
the box because most of the Cypress commands only query the DOM which does not produce Trusted Types
violations.

Even though Cypress supports Trusted Types out of the box, there are a few nuances the developer has
to overcome to be able to test the application and Trusted Types violations. For this reason, we
created a Cypress plugin that abstracts the low-level details and provides a nicer API for the
developers to use.

\subsection{Applying the CSP header}

Cypress removes the CSP header sent by the application server to load the application in an
\emph{iframe}, which could otherwise be prevented by a \emph{scriptSrc} CSP directive. There is a
workaround for now, which is to intercept the request of the initial HTML application payload and
copy the CSP policy from the response headers to the response body inside an HTML meta tag.
\cite{cypress:csp_removal_issue}.

This is tedious for the developers, so we created a custom command in our plugin which does
exactly this.

\bigskip
\begin{lstlisting}[language=JavaScript, caption=Intercept requests and enable CSP header inside via meta tag]
// NOTE: Based on https://glebbahmutov.com/blog/testing-csp-almost/
Cypress.Commands.add('enableCspThroughMetaTag', (options) => {
  const { urlPattern } = options ?? {};

  // Intercept all requests by default
  cy.intercept(urlPattern ?? '**/*', (req) => {
    return req.reply((res) => {
      const csp = res.headers['content-security-policy'];
      if (!csp || typeof res.body !== 'string') return;

      res.body = res.body
        .replace(
          new RegExp('<head>([\\s\\S]*)</head>'),
          new RegExp(`<head><meta http-equiv="Content-Security-Policy" content="${csp}">$1</head>`).toString()
        )
        // The following are needed because the regex replacement above inserts some characters
        .replace('/<head>', '<head>')
        .replace('<\\/head>/', '</head>');
    });
  }).as('enableCspThroughMetaTag');
});
\end{lstlisting}

\subsection{Testing the violations}

While it is easy to test whether Trusted Types are supported (\ref{lst:cypress_test_tt_support}) it
is harder to test whether some part of the code triggered Trusted Types violation. The reason is
that the violation does not produce any user-visible behavior because the DOM is unchanged and an
uncaught exception is thrown and Cypress automatically fails the test. The developer needs to listen
to \emph{uncaught:exception} handler from Cypress and explicitly recognize a Trusted Types
violation based on the error thrown and tell Cypress that this error is intended so it does not fail
the test.

% Add space to the options start on a new page
\vspace{25mm}
\bigskip
\begin{lstlisting}[language=JavaScript, caption=Test Trusted Types support, label={lst:cypress_test_tt_support}]
it('supports TT', () => {
  expect(window.trustedTypes).not.to.be.undefined;
});
\end{lstlisting}
% To make the page look a bit bigger
\vspace{3mm}

To simplify this low-level handling for the developers we created a custom command which recognizes
Trusted Types violations and remembers them for the lifetime of the current test so they can be
asserted later on.

% To make the page look a bit bigger
\vspace{3mm}
\bigskip
\begin{lstlisting}[language=JavaScript, caption=Custom command to catch Trusted Types violations]
Cypress.Commands.add('catchTrustedTypesViolations', () => {
  if (catchTrustedTypesViolationsEnabled) return;
  catchTrustedTypesViolationsEnabled = true;
  cy.clearTrustedTypesViolations();

  // https://docs.cypress.io/api/events/catalog-of-events#To-catch-a-single-uncaught-exception
  cy.on('uncaught:exception', (err) => {
    const type = violationTypes.find((type) => err.message.includes(quote(type)));
    if (type) {
      trustedTypesViolations.push({
        type,
        message: extractViolationMessage(err),
        error: err,
      });
      // Return false to prevent the error from failing this test
      return false;
    }
  });
});
\end{lstlisting}
% To make the page look a bit bigger
\vspace{3mm}

The final part of the plugin is the API to assert the caught violations. There are multiple commands
which assert that certain types of violations happened. An example of a full Cypress test which
asserts the violation is listed in (\ref{lst:cypress_test}).

% Add space to the options start on a new page
\vspace{50mm}

\bigskip
\begin{lstlisting}[language=JavaScript, caption=Example Trusted Types violation test, label={lst:cypress_test}]
it('assertTrustedTypesViolations', () => {
  cy.contains('unsafe html').click();
  cy.contains('unsafe html').click();
  cy.contains('duplicate policy').click();
  cy.contains('unsafe script').click();

  cy.assertTrustedTypesViolations([
    {
      type: 'TrustedHTML',
      message:
        "Failed to set the 'srcdoc' property on 'HTMLIFrameElement': This document requires 'TrustedHTML' assignment.",
    },
    {}, // No assertion is made for this violation
    {
      type: 'TrustedTypePolicyFactory',
      message: `Failed to execute 'createPolicy' on 'TrustedTypePolicyFactory': Policy with name "my-policy" already exists.`,
    },
    { type: 'TrustedScript' },
  ]);
});
\end{lstlisting}

\subsection{Releasing the plugin}

The plugin is implemented in a standalone repository \cite{cypress-trusted-types:github} and
published as an npm (\ref{def:npm}) package \cite{cypress-trusted-types:npm} for anyone to use.
Together with the plugin API there is an example application and tests which showcase the plugin in
action. The plugin is to be added to the cypress community list of plugins for increased visibility.
