\chapter{Trusted Types integration process}
\label{tt_integration_setup}

Integrating Trusted Types to target applications and libraries might seem as a complex task that
requires good knowledge of the target inner workings. There is a small sample of already implemented
integrations, which shows that the necessary code changes are relatively small
\cite{tt_integration_list}. Based on our experience, the integration does not require an exprteese
of the projects inner workings neither. This experience is also supported by other integration
authors \cite{tt_web_framework_paper}. Each integration is different, but on a high level they
follow the same steps:

\begin{enumerate}
  \item Locate all of the DOM sinks
  \item Find the most suitable workaround for every sink found
  \item Implement the integration and release a new version of the target
\end{enumerate}

\section{Locate all of the DOM sinks}

Locating the sinks of the integration target is important for scoping the integration effort and the
implementation afterwards. This task is complex since there is no bulletproof way of locating all of
the sinks in a codebase. This is because JavaScript is a dynamic language and it is possible to
access and set the sink values dynamically using the property element access.

Fortunately, there are a few methods and tools, which can help to catch most of the sinks:

\begin{enumerate}
  \item Static search through the codebase
  \item Static code analyzers
  \item Using the target in an real world application
\end{enumerate}

\subsection{Static search through the codebase}

This method is the simplest one and the least effective. Statically searching for sinks throughout
the codebase produces many false positives and also false negatives. The former are produced for
read only sink usage, the latter when actual sinks are missed due to implementation limitations. The
output of the static search is usually cluttered with less important violations in tests and built
tools, which you need to manually exclude from the search.

Nevertheless, this method is easy to reason about, fast to iterate and the final search results can
provide a good estimate of the integration scope. This works well in practice since there is an
assumption that the target source code is not malicious and doesn't actively use code patterns which
are trying to hide or cause an XSS vulnerability.

There is not an actively maintained static search tool for finding DOM sinks. However, it is
relatively easy to build a script, which uses the existing tools such as \textit{grep} with a list
of already known sinks \cite{xss_sink_finder}.

\subsection{Static code analyzers}

A better approach at searching for the DOM sinks statically, is to search through the AST of the
code. The quality of the output produced depends on the quality of the AST information. When the
codebase is using TypeScript, the AST information is generally richer compared to codebases using
JavaScript. One can then build and use tools like Tsec \cite{tsec_github} which uses the compiler to
parse the source code and create the AST to ultimately find the sinks in a more reliable manner and
produces less false positives.

These tools have an advantage that they can be used to maintain the Trusted Types compatibility even
after the violations are fixed. For example, they can be run in CI pipeline or their output can be
leveraged by linters and other language plugin tools and the errors can be shown to the developers
early on and directly in their IDE \cite{tsec_lsp}.

\subsection{Using the target in a real world application}

While static search through the codebase is fast and locates most of the sinks, it usually doesn't
catch them all. If the integration target is a library, the integration author should verify the
integration implementation on the applications using the target as a dependency. Also, having good
test coverage in the target code greatly increases the chance that the integration is
implemented correctly.

That said, finding a suitable application might be challenging and it may not be an ideal way to
test all edge cases. For example, when implementing an integration to React, one can find lot of
real world applications using React, but most of them are already XSS free so you can only verify if
the integration doesn't break these clients.

One recommendation for the clients are to use the the report only mode of Trusted Types CSP. This
allows applications to migrate their code gradually and allows them to focus on application
features.

Library authors need to test how the integration works when used with applications that uses the
DOM sinks and now has to produce Trusted Types values. These are done best by creating an
application from scratch by the integration author.

\section{Find the most suitable workaround for every sink found}

In this section, we assume that all of the sinks discussed produce Trusted Types violations. Note,
that a sink can produce a Trusted Types violation for a value that is probably safe. For example, this happens
when a constant string is assigned to an innerHTML property of an DOM element.

Generally, there are three ways how to resolve a Trusted Types violation produced by a sink:

\begin{enumerate}
  \item Refactor the code not to use the sink value
  \item Wrap the value passed to the sink in a Trusted Types policy
  \item Ensure Trusted value is not altered before reaching the sink
\end{enumerate}

\subsection{Refactor the code not to use the sink value}

For some dangerous sinks, there is a safer alternative that can be used. For example, for non script
elements, one can use `element.innerText` instead of `element.innerHTML`. Another way is to create
DOM nodes manually and only append the generated elements to the target element as child.

This method is only available when there is a safer API alternative, which is often not the case and
this solution is not applicable.

\subsection{Wrap the value passed to the sink in a Trusted Types policy}

When there is no way to avoid using the sink and the value is trusted, one has to use a Trusted
Types policy to promote the trusted value to a Trusted Types instance. One should apply the Trusted
Types policy where the trusted value is created, not where it is passed to a sink. For example, the
author should promote a value to Trusted Type immediately after sanitization or escaping. This way
the immutability of Trusted Types can preserve the "trust" independently of how the value flows
through code until it reaches the sink.

The easy part of this method is wrapping a value inside a policy, but if the application
or library used to modify this value it now needs to be refactored to readonly usage of these values
(\ref{tt_integration_ensure_tt_not_altered}).

\label{tt_integration_ensure_tt_not_altered}
\subsection{Ensure Trusted value is not altered before reaching the sink}

As we mentioned, Trusted Types instances are immutable values. Previously the DOM sinks accepted a
string values so the application or library might be used to call string operations on the value.
Such code was valid, but now the Trusted Types values do not have string methods defined and trying
to call them will result in an error. Another problem is stringification which converts the Trusted
Types instance to a string, which then throws an error when the value is passed to the sink.

The integration author needs to refactor the library or application to prevent modifying the
Trusted values to preserve the trust guarantees secured by Trusted Types.

\section{Implementing and releasing the integration}

Implementing the integration and releasing a new version of the target is undoubtedly an important part
of Trusted Types integration. However, this can take long time for multiple reasons, which are
described in separate sections:

\begin{enumerate}
  \item Reasoning about the integration
  \item Trusted Types compatibility in dependencies
  \item Knowledge required for the integration author
\end{enumerate}

\subsection{Reasoning about the integration}
\label{sub:reason_about_integration}

Proving that the integration is correct means to determine if all sinks have been located and
properly addressed. This is generally infeasible or impossible to determine. Project usually look at
patches empirically. If the integration looks correct and is properly tested then it is safe to
assume that the integration is indeed correct. This is especially true when the integration is
tested by large scale organizations by lot of services.

That being said, Trusted Types enforcement is a breaking change and project should make this change
opt-in or release a new version with breaking changes \cite{dom_purify_major_version}. If the
integration is turned on by default when Trusted Types are available in the browser, the target is
risking breaking existing applications \cite{dom_purify_breakage}.

Releasing an opt-in change might be problematic if the target is not configurable. This can result
in the integration being put behind a feature flag which hurts the adoption
\cite{react_tt_feature_flag}.

\subsection{Trusted Types compatibility in dependencies}
\label{sub:tt_compatibility_in_deps}

Implementing the integration in large projects, which consists of many dependencies can bring a lot
of overhead, because to be fully Trusted Types compliant one must ensure that all of the
dependencies are Trusted Types complaint as well.

When there is non compliant dependency, one has multiple options:

\begin{itemize}
  \item Implement the integration for the dependency - This option leads to recursive integrations
        and has all the problems already mentioned and can possible lead to many more integrations.
  \item Find an alternative dependency - This option is often impossible as a viable alternative
        might not exist and projects might not want to use an alternative only because of Trusted
        Types compliance.
  \item Use Trusted Types default policy - This option should only be used by end applications and
        is more complex. The application is not fully Trusted Types compliant and it can also lead
        to a reduced performance.
\end{itemize}

\subsection{Knowledge required for the integration author}
\label{sub:trust_integration_author}

The integration author can neither be a security engineer familiar with Trusted Types and neither a
part of the engineering team of the target familiar with the target code. This means that the author
might not be fully familiar with both of the technologies and the respective teams, which can result
in both:

\begin{itemize}
  \item Increased propability of a bug in implementation
  \item Harder and longer review, since the reviewers might also lack the information about Trusted
        Types or simply ignore the PR due to other priorities
\end{itemize}
