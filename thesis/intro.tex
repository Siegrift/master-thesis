\chapter{Introduction}

\section{Trusted Types}

Trusted Types is a relatively modern web API designed by Google based on a long history of
mitigating XSS \cite{tt_design_history}.

It is a browser security feature that limits access to dangerous DOM APIs to protect against DOM XSS
(\ref{def:dom_xss}). Trusted Types provide type guarantees to all frontend code by enforcing
security type checks directly in the web browser. They are delivered through a CSP (\ref{def:csp})
header and have a report-only mode that does not change application behavior and an enforcement mode
that may cause user-observable breakages \cite{tt_background}.

When enforced, Trusted Types block dangerous injection sinks (\ref{def:dom_source_sink}) from being
called with values that have not passed through a Trusted Types policy \cite{tt_background}.

There are many other resources which can be used to explore Trusted Types in depth
\cite{tt_resources}.

\section{Web frameworks and libraries}

Web frameworks and libraries is a software that is designed to support development of web
applications and services related to web applications. Frameworks try to solve common problems faced
in web development, such as building and reusing user interfaces, managing state, performing API
requests and enforcing the best security practices.

\subsection{Next.js}
\label{intro-nextjs}

Next.js is one of the most popular frameworks for building web applications. The framework is build
upon React library which is the most used library as of 2021 \cite{react_most_used_2021}. There are
many reasons for why Next.js is among the most popular frameworks. Notably, it is opinionated about
how the application backend should work. This design choice allows Next.js to decide the best
possible way to build a particular web page. The important fact to take away is that Next.js is not
only about the client side (the DOM), but it can also handle server side rendering, data processing,
API requests which open up different means of attacks such as reflected XSS, SQL injections and
more.

Next.js is a complex framework when looking at the source code. As of Oct 6, 2021 it consists of
more then 200,000 lines of code. Looking at each file or have a complete understanding of the
entire codebase is impossible for external contributor.
