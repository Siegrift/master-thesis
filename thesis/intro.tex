\chapter{Introduction}

In this chapter we give a brief overview on one of the most common web application vulnerabilities
(Cross site scripting) and then explain the design of Trusted Types and how it helps to mitigate
this vulnerability. Lastly, we describe the frameworks and libraries we work with in the thesis and
the motivation for our work.

\section{Cross site scripting (XSS)}

Cross site scripting (XSS) is one of the most prevalent vulnerabilities on the web. It is an attack
of web applications taking untrusted user input and interpreting is as code without sanitization or
escaping. The XSS has many subcategories, for this paper the most important is DOM XSS
(\ref{def:dom_xss}), where the injection happens on the client side, the browser
\cite{tt_web_framework_paper}.

This vulnerability became more widespread with the boom of single page applications where most of
the behaviour is achieved by modifying the DOM using JavaScript. There are many functions, element
attributes and properties in the DOM API which interpret the arguments as executable code. We call
these DOM sinks (\ref{def:dom_source_sink}). These sinks make it easy for developers to accidentally
introduce this vulnerability \cite{tt_web_framework_paper}.

\bigskip
\begin{lstlisting}[language={}, caption=Examples of DOM XSS attack vectors \cite{dom_xss_portswigger} \cite{tt_web_framework_paper}]
document.write()
document.writeln()
document.domain
element.innerHTML
element.insertAdjacentHTML
element.onevent
DomParser.parseFromString
frame.srcdoc
eval()
script.src
Worker()
\end{lstlisting}

One of the most basic examples of XSS is the interpolation of URL parameters in the DOM. The
attacker can prepare a dangerous URL which he sends to a victim. The victim executes the payload
just by navigating to the site sent by the attacker.

\bigskip
\begin{lstlisting}[language=HTML, caption=Basic example of XSS via unsafe URL parameter interpolation]
<!--
Assume this page is on https://example.com.
It can be misused by the following attack payload:
https://example.com?<img%20src=x%20onerror="alert(1)"></img>
-->
<!DOCTYPE html>
<html lang="en">
  <body>
    <div id="content"></div>
    <script type="text/javascript">
      const content = decodeURIComponent(location.search.substr(1))
      document.getElementById('content').innerHTML = 'URL content: ' + content
    </script>
  </body>
</html>
\end{lstlisting}

The consequences of XSS vary a lot. Their severity can range from benign annoyances to catastrophic
user consequences such as full account compromise, disclosure of user's cookies, storage, secrets or
session. Other attacks may use XSS to change the application content and present falsy information
to the user.

There are many attempts to reduce the risk of DOM XSS either by dynamic or static checkers. The
former usually suffer from performance and scalability issues when applied on large codebases while
the latter provide suboptimal results due to JavaScript dynamic nature
\cite{tt_web_framework_paper} \cite{owasp_xss_cheatsheet}.

\section{Trusted Types}

Trusted Types is a relatively modern web API designed by Google based on a long history of
mitigating XSS \cite{tt_design_history}.

It is a browser security feature that limits access to dangerous DOM APIs to protect against DOM
XSS. Trusted Types provide type guarantees to all frontend code by enforcing security type checks
directly in the web browser. They are delivered through a CSP header and have a report-only mode
that does not change the application behavior and an enforcement mode that may cause user-observable
breakages \cite{tt_background}.

When enforced, Trusted Types block dangerous injection sinks (\ref{def:dom_source_sink}) from being
called with values that have not passed through a Trusted Types policy \cite{tt_background}. If an
untrusted value is passed to sink a Trusted Types violation is raised and the DOM is unaffected. In
practice this means that a potential DOM XSS has been prevented.

There are many other resources which can be used to explore Trusted Types in detail
\cite{tt_resources}.

\subsection{Threat model}

Trusted Types is a powerful API with a well scoped threat model. The main goals of the API are to
\cite{tt_spec:goals}:

\begin{itemize}
  \item reduce the risk of client side vulnerabilities caused by injection sinks.
  \item replace the insecure by default APIs with safer alternatives which are harder to misuse.
  \item encourage a design where the code affecting the application security is encapsulated in a
        small part of an application.
  \item reduce the security review surface for applications and libraries.
\end{itemize}

The main idea behind Trusted Types is to replace the dangerous APIs with safer alternatives. It is
very tempting to extend the threat model of Trusted types to cover other surfaces such as various
server side attacks. However, this is a very complex area and it is part of the non goals of Trusted
Types \cite{tt_spec:non_goals}. Some of the non goals include:

\begin{itemize}
  \item preventing or mitigating server side generated markup attacks. Defending against XSS on both
        client and server can be really complex, especially for applications where parts of the code
        can run on both client and server, such as NextJS (\ref{intro-nextjs}). To address these
        attacks, use the existing recommended solutions like templating systems or CSP script-src.
  \item controling subresource loading. Trusted Types deal with code running in realm of the current
        document and does not guard subresources.
  \item guarding cross origin JavaScript execution, for example loading new documents via data: URLs
  \item protecting against malicious developers of the web application.
\end{itemize}

\subsection{Content Security Policy (CSP)}
\label{csp}

Trusted Types are enabled through a CSP (\ref{def:csp}) using two different directives:

\begin{itemize}
  \item \textit{require-trusted-types-for} -- This directive instructs user agents to control the
        data passed to DOM XSS sink functions (\cite{mdn:require-trusted-types-for}).

        \bigskip
        % NOTE: To remove vertical space when using listing without caption \begin{lstlisting}[language={}, belowskip=-1.2 \baselineskip]
        \begin{lstlisting}[language={}, caption=Syntax of require-trusted-types-for directive]
Content-Security-Policy: require-trusted-types-for 'script';\end{lstlisting}

  \item \textit{trusted-types} -- This directive instructs user agents to restrict the creation of
        Trusted Types policies (\cite{mdn:trusted-types}). Syntax:

        \bigskip
        \begin{lstlisting}[language={}, caption=Syntax of trusted-types directive]
Content-Security-Policy: trusted-types;
Content-Security-Policy: trusted-types 'none';
Content-Security-Policy: trusted-types <policyName>;
Content-Security-Policy: trusted-types <policyName> <policyName> 'allow-duplicates';\end{lstlisting}

\end{itemize}

These directives together enable and configure Trusted Types behaviour for the particular web
application. They allow the application authors to define rules guarding write access to the DOM
sinks and thus reducing the DOM XSS attack surface to small, isolated parts of the web application
codebase, facilitating their monitoring and manual code review.

Apart from the standard \textit{Content-Security-Policy} header there is also
\textit{Content-Security-Policy-Report-Only} which can be used to enable Trusted Types in report
only mode. This mode interprets Trusted Types violations only as warnings. This way the application
can gradually work on Trusted Types compliance without breaking the existing application features.
It is also recommended to use the report only mode in production for some time to make sure the
integration is working as expected \cite{tt_web_framework_paper}.

Once the Trusted Types are enabled, the browser changes the behaviour of insecure DOM API sinks and
expects "trusted" values instead of regular strings. These trusted values are created via Trusted
Types policies.

\subsection{Trusted Types policies}
\label{subsec:tt_policy}

The core part of Trusted Types API are policies which are factory functions for creating "trusted"
values which can be assigned to DOM sinks when Trusted Types are enabled. The policies are created
by the application and access to them should be restricted. In javascript this can be easily
achieved by creating a policy in its own module and exporting only a very specific functions which
use this policy internally.

\bigskip
\begin{lstlisting}[language=JavaScript, caption=Creating Trusted Types policy, label={lst:create_tt_policy}]
const identity = (id) => id
const createHTMLCallback = identity;
const createScriptCallback = identity;
const createScriptURLCallback = identity;
const myPolicy = window.trustedTypes.createPolicy('my-policy', {
  createHTML: createHTMLCallback,
  createScript: createScriptCallback,
  createScriptURL: createScriptURLCallback,
});
\end{lstlisting}

\bigskip
\begin{lstlisting}[language=JavaScript, caption=Create trusted value using a policy]
const trustedHtml = myPolicy.createHTML("<span>safe html</span>");
\end{lstlisting}

The code listing \ref{lst:create_tt_policy} creates a Trusted Types policy using the callback
functions. This callback function is called when the policy is used to create a trusted value. It
receives the sink value of string type as an argument and has to return a string value that is XSS
free and thus can be trusted. In practice, this function may be implemented as identity if the
payload is already trusted. Another good example would be to sanitize the sink value inside this
callback. In case the payload should not be used, you can return \textit{null} or \textit{undefined}
which will trigger a Trusted Types violation.

\bigskip
\begin{lstlisting}[language=JavaScript, caption=Using a policy to sanitize HTML values]
const myPolicy = window.trustedTypes.createPolicy('sanitize-html', {
  createHTML: (untrustedValue) => DOMPurify.sanitize(untrustedValue),
});
\end{lstlisting}

\subsection{Default policy}

There is one special case for Trusted Types policies. Applications may create a policy called
"default". This policy has a special behaviour. When a string value is passed to a DOM sink when
Trusted Types are enabled, the user agent will implicitely flow the untrusted string value through
the default policy. The callback function of the default policy receives three arguments instead of
one -- the string payload, sink type and sink name respectively. This allows the application to
recover from an unexpected sink usage, for example by sanitizing the untrusted value. If the default
policy does not exist or returns \textit{null} or \textit{undefined} a CSP violation will be
triggered \cite{tt_spec:default_policy}. Applications can use this policy to enable enforcement mode
even though the application is not fully Trusted Types complaint.

This feature is intended to be used by applications with legacy code that uses injection sinks. The
callback functions of the policy should be defined with very strict rules to prevent bypassing
security restrictions enforced by Trusted Types API. For example, having an "accept all" default
policy defeats the whole purpose of Trusted Types API. Developers should be very cautious when using
the default policy and preferably use it only for a transition period until the legacy code is
refactored not to use the dangerous DOM sinks \cite{tt_spec:default_policy}.

\bigskip
\begin{lstlisting}[language=JavaScript, caption=Creating a default policy \cite{tt_spec:default_policy}]
trustedTypes.createPolicy('default', {
  createScriptURL: (value, type, sink) => {
    return value +
      '?default-policy-used&type=' +
      encodeURIComponent(type) +
      '&sink=' +
      encodeURIComponent(sink);
  }
});
\end{lstlisting}

\subsection{Reviewability}

Consider the problem of security reviews for web applications. There are many tools and methodologies
which can help reason about the application security, but there is no automated way that can assert
the application safety. This means that it is still necessary for security engeneers to manually
review the implementation and look for potential vulnerabilities and analyze them.

When focusing on client side XSS, the engineer has to determine whether application uses dangerous
DOM sinks and whether there is way for attacker to misuse them.

\bigskip
\begin{lstlisting}[language=JavaScript, caption=Possibly dangerous function]
function setHtml(element, html) {
  element.innerHTML = html;
}
\end{lstlisting}

Is the function in the listing above safe? There is not enough information to answer this. The
safety of the function depends on the context and how it is used. More generally, the safety of a
function depends on both its direct and indirect callers, both present and future ones
\cite{tt_design_history}.

\bigskip
\begin{lstlisting}[language=JavaScript, caption=Usage of the possibly dangerous function, label={lst:dangerous_fn_usage}]
  function processHtml(html) {
    setHtml(document.body, html);
  }

  function processUserData(data) {
    log('Processing user data');
    // Is this safe? Can this call change "data.html"?
    someThirdPartyCall(data);
    processHtml(data.html);
  }
\end{lstlisting}

Looking for the callers of the function can be difficult because of the dynamic nature of
JavaScript. Also, JavaScript is a mutable language which makes it harder to reason about function
calls, especially the third party ones.

When the application enforces Trusted Types, the engineer doesn't have to care about the dangerous
functions and its callers. The focus of the security review shifts to reasoning about the creation
of trusted values and policies. Once a trusted value is created it is immutable and the policy which
created it provides the security guarantees. When a trusted value reaches a sink, this guarantee
still holds independently of how was the value passed around throughout the callers. Consider the
third party call in the listing \ref{lst:dangerous_fn_usage} and assume \textit{data.html} contains
a TrustedHTML value. If the third party call modified this value a violation would be thrown when
passed to the DOM sink. It could also replace the trusted value with a different TrustedHTML value.
However, the new value is again trusted, so this is not a security vulnerability, but an application
bug.

\subsection{Browser support and polyfill}

Trusted Types are currently supported in Chromium family of browsers \cite{mdn:tt_compatibility}.
Developers creating Trusted Types compliant application should always check if Trusted Types are
available in the current execution context, which does not necessary need to be a browser. It is
very common for nodeJS applications to pre render the client side code on the server to provide
faster experience for the end users. This brings additional complexity and new attack vectors in
forms of reflected XSS, various kinds of injection and more, which are out of the threat model under
which Trusted Types operate.

The nice thing is that once the application is Trusted Types compliant then all of the application
sinks are protected. The application can also use a polyfill for browsers which do not support
Trusted Types, which results in the same security properties as in the browsers where they are
supported \cite{xss_nowhere_with_polyfill}.

Browsers which do not support Trusted Types yet can use one of the variants of the polyfill:

\begin{itemize}
  \item Full -- Defines the Trusted Types API, parses the \textit{meta} tag from the HTML and
        enables the enforcement in the DOM.
  \item API only -- Defines the Trusted Types API so you can use policies and create trusted values,
        but no enforcement rules are applied.
  \item Tiny -- Polyfills only the most important part of the API surface with a single line of
        code for minimal bundle size.
\end{itemize}

\section{Web frameworks and libraries}

Web frameworks and libraries is a software that is designed to support development of web
applications and services to improve developer experience. They try to solve common problems faced
in web development, such as building user interfaces, testing, building and bundling the
application. Typical web application consists of numerous libraries and supporting frameworks which
together create the resulting web application.

\subsection{Next.js}
\label{intro-nextjs}

Next.js is one of the most popular frameworks for building web applications. The framework is build
upon React which is the most used library as of 2021 \cite{react_most_used_2021}. The design of
Next.js allows building the application on a page by page basic to maximize the user experience.
Some parts of the application can be statically rendered, some rendered server side or fully client
side. Next.js is thus not only about the client side (the DOM), but it can also handle server side
rendering, data processing, API requests which open up different means of attacks such as reflected
XSS, SQL injections and more.

\subsection{Create React App}
\label{intro-cra}

Create React App is an officially supported way to create single-page React applications. It offers
a modern build setup with no configuration. It is a CLI which creates and configures a minimal React
application. This is ideal for starting new projects as the configuration is abstracted away from
the user.

The source code of this tool does not depend on React directly. It only generates the project files
using based on hard coded template and then it installs the latest version of React and other
necessary dependencies.

\subsection{Cypress}
\label{intro:cypress}

Cypress is one of the most popular frameworks for end to end testing of web applications
\cite{js_state:testing}. It enjoys a rich ecosystem of plugins and supporting software. It runs the
tests in real browsers and is fast and reliable due to its unique architecture. Developers can test
the Trusted Types compliant applications in Cypress out of the box, because most of the Cypress
commands only query the DOM which does not produce Trusted Types violations.

\section{Motivation and background}

Trusted Types provide opportunity to significantly reduce the risk of client side XSS and has been
proven in various scale projects and libraries \cite{tt_web_framework_paper}
\cite{tt_integration_list}. However, projects need to refactor parts of their code, which can be
difficult, especially in the open source world where the software is composed by multiple
dependencies which can't easily be modified. This presents a large barrier in Trusted Types adoption
\cite{tt_web_framework_paper} with a "chicken and egg" problem. There is not enough pressure for
library authors to migrate to Trusted Types because there is not is not enough usage and there is
not enough usage because a lot of applications are prevented to migrate because their dependencies
do not work with Trusted Types.

In this paper we try to analyze some of the most popular open source frameworks and libraries and
show that there are workarounds for projects to use Trusted Types without too much hassle. We
also stress the need for more tooling and libraries with Trusted Types support, where we see a big
opportunity for open source contributors to engage.

There is usually a knowledge gap between security engineers and software developers, the former
focus on security, the latter on the application features. Trusted Types bridge this gap by
providing secure by default software. The ideal scenario would be if the open source libraries used
Trusted Types transparently to the application authors.
